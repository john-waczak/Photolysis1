% Please refer README file for more details about this document

\documentclass[twoside,twocolumn]{article}

\usepackage{tabulary,graphicx,times,caption,fancyhdr,amsfonts,amssymb,amsbsy,latexsym,amsmath}
\usepackage[utf8]{inputenc}
\usepackage{url,multirow,morefloats,floatflt,cancel,tfrupee,textcomp,colortbl,xcolor,pifont}
\usepackage[nointegrals]{wasysym}
\urlstyle{rm}

\makeatletter

%Etal definition in references
\usepackage{ifxetex}
\ifxetex\else
  \usepackage{dblfloatfix}
\fi

\@ifundefined{subparagraph}{
\def\subparagraph{\@startsection{paragraph}{5}{2\parindent}{0ex plus 0.1ex minus 0.1ex}%
{0ex}{\normalfont\small\itshape}}%
}{}

\def\URL#1#2{\@ifundefined{href}{#2}{\href{#1}{#2}}}

%%For url break
\def\UrlOrds{\do\*\do\-\do\~\do\'\do\"\do\-}%
\g@addto@macro{\UrlBreaks}{\UrlOrds}

\makeatother
\def\floatpagefraction{0.8} 
\def\dblfloatpagefraction{0.8}

%%%%%%%%%%%%%%%%%%%%%%%%%%%%%%%%%%%%%%%%%%%%%%%%%%%%%%%%%%%%%%%%%%%%%%%%%%

\usepackage[paperheight=11in,paperwidth=8.3in,margin=2.5cm,headsep=.7cm,top=2.5cm]{geometry}
\usepackage[T1]{fontenc}
\def\floatpagefraction{0.8}
\widowpenalty 10000
\clubpenalty 10000

\renewenvironment{abstract}
	{\trivlist\item[]\leftskip0pt\par\vskip4pt\noindent
  	\textbf{\abstractname}\mbox{\null}\\}
	{\par\noindent\endtrivlist}

\def\keywords#1{\par\medskip\par\noindent\textbf{Keywords}: #1\par}

\linespread{1.13} \date{} \emergencystretch 8pt

\captionsetup[figure]{labelfont=normal,skip=1.4pt,aboveskip=1pc}
\captionsetup[table]{labelfont=normal,skip=1.4pt}

\makeatletter
\def\author#1{\gdef\@author{\hskip-\tabcolsep%
	\parbox{\textwidth}{\raggedright\bfseries#1\\[1pc]}}}
\def\address[#1]#2{\g@addto@macro\@author{\\\hskip-\tabcolsep\parbox{\textwidth}{\raggedright%
	\normalsize\normalfont\textsuperscript{#1}#2}}}
\let\addresslink\textsuperscript
\def\correspondence#1{\g@addto@macro\@author{\\\hskip-\tabcolsep\parbox{\textwidth}{\raggedright%
	\vspace*{10pt}\normalsize\normalfont~\\#1~\\[12pt]}}}
\def\email#1{\g@addto@macro\@author{\\\hskip-\tabcolsep\parbox{\textwidth}{\raggedright%
	\normalsize\normalfont Emails: #1}}}

\def\title#1{\gdef\@title{\vspace*{-30pt}%
	\raggedright\textbf{\@journaltitle}~\\%
  \raggedright\bfseries\ifx\@articleType\@empty\vspace*{20pt}\else%
  \vspace*{20pt}\@articleType\vspace*{20pt}\\\fi#1}}
\let\@journaltitle\@empty \def\journaltitle#1{\gdef\@journaltitle{{\normalfont\itshape#1}}}
\let\@articleType\@empty \def\articletype#1{\gdef\@articleType{{\normalfont\itshape#1}}}

\let\@runningHead\@empty \def\RunningHead#1{\gdef\@runningHead{{\normalfont #1}}}

\usepackage{fancyhdr}
\fancypagestyle{headings}{\renewcommand{\headrulewidth}{0pt}\fancyhf{}
  \fancyhead[R]{\itshape\@runningHead}
  \fancyfoot[C]{\thepage}}
\pagestyle{headings}

\fancypagestyle{plain}{\renewcommand{\headrulewidth}{0pt}%
	\fancyhf{}\fancyhead[R]{Indoor Air}
  \fancyfoot[C]{\thepage}}
\makeatother

\usepackage[%
	numbers,sort&compress%
	%authoryear
  ]{natbib}

\setcounter{secnumdepth}{0}
\usepackage{float,xcolor}

\journaltitle{Indoor Air}
\articletype{Research Article} % Research Article/Review Article/Clinical Study

\begin{document}

% Title of the document
\title{Comprehensively Calculating Indoor Photolysis Rates}

% Author names
\author{%
John Waczak\addresslink{1, 3},
David J. Lary\addresslink{1, 2, 3},
Matthew D. Lary\addresslink{1, 3} 
John Sadler\addresslink{3}, 
Andrew Redmond\addresslink{3},
Joseph Urso\addresslink{3} and 
Amy Carenza\addresslink{3} 
}
		
% Affiliation
\address[1]{Hanson Center for Space Sciences, University of Texas at Dallas, Richardson TX 75080, USA}
\address[2]{Complex Exposure Threats Center Network (CETC), U.S. Department of Veterans Affairs}
\address[3]{ActivePure Technologies LLC, 14841 Dallas Pkwy #500, Dallas, TX 75254}

% Corresponding author details
\correspondence{Correspondence should be addressed to John Waczak: John.Waczak@utdallas.edu}

% Emails of authors
\email{John.Waczak@utdallas.edu (John Waczak)}%

% Running Head
\RunningHead{Comprehensively Calculating Indoor Photolysis Rates}

\maketitle 

% Abstract
\begin{abstract}
Abstract text goes here 


% Keywords - if any
\keywords{Key1; Key2; Key3}
\end{abstract}
    
% First level heading
\section{Introduction}

In the global outdoor environment, sunlight incident on the Earth's atmosphere is the ultimate driving force of a host of photo-chemical and environmental processes (e.g. photosynthesis, photolysis, atmospheric radiative heating). Accurately characterizing atmospheric radiative transfer plays a key role in modeling outdoor atmospheric chemistry and the weather/climate system (e.g. \cite{Chandrasekhar1960, Lenoble1985, Lary1991a, Lary1991b, Deutschmann:2011, Hartmann:2016, Buehler:2017, Zhang:2019}). Likewise, light is a key driver for indoor air quality (iAQ), particularly when this involves the high-energy photons associated with the ultraviolet wavelengths (\hbox{10--400 nm}, with \hbox{UV-A 315--400 nm}, \hbox{UV-B 280--315 nm}, \hbox{UV-C 100--280 nm}). 

The photolysis of abundant iAQ gases such as O$_2$, H$_2$O, and O$_3$ lead to the production of highly reactive free radical oxidizing agents such O($^1$D), O($^3$P), and OH and the resultant initiation of large photochemical reaction cascades that have an enormous impact on iAQ. Further, if nitrogen oxides are present, then oxidizing agents such as NO$_3$ will also be produced by photolysis. Likewise, if halogens are present, such as due to the use of cleaning products involving bleach, then oxidizing agents such as Cl will also be produced by photolysis. Each of these oxidizing agents plays a central role in iAQ. The precise impact is a sensitive function of the local physical and chemical environment. Given that thousands of VOCs can be present in iAQ we begin to see just how daunting comprehensively characterizing iAQ becomes.  

Further, the oxidation of VOCs can lead to the formation of secondary organic aerosols, typically ranging in size from 0.1 to 1 $\mu$m. The size of aerosols determines their depth of penetration into the human respiratory tract and lungs. Generally, larger particles tend to deposit in the upper airways (nasal cavity, pharynx, larynx), while smaller particles have a greater likelihood of reaching the lower airways (trachea, bronchi, bronchioles, alveoli). Coarse particles (particles with aerodynamic diameters between 2.5 to 10 $\mu$m) primarily deposit in the upper airways, with limited penetration into the lower airways. Fine particles (particles with aerodynamic diameters between 0.1 and 2.5 $\mu$m) can penetrate deeper into the respiratory tract, with a significant portion reaching the lower airways, including bronchi and bronchioles. Ultrafine particles (particles with aerodynamic diameters less than 0.1 micrometer), such as SOA, have high potential for deep penetration into the lower airways and alveolar region of the lungs.




Unlike passive systems, active systems neutralize pathogens while they are still in the air. It is therefore imperative that when such active systems are deployed the detailed impacts on iAQ are fully characterized. 



\bibliographystyle{hindawi_bib_style}
\bibliography{references.bib} % without .bib extension

\end{document}
